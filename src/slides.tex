%\documentclass{beamer}
\documentclass[14pt,handout]{beamer}
\usetheme{Darmstadt}
\usepackage{graphicx}
\usepackage{hyperref}
\usepackage[german]{babel}
\usepackage[T1]{fontenc}
\usepackage[utf8]{inputenc}
\setbeamertemplate{footline}[frame number]

\usepackage{pdfcomment}
\newcommand{\ben}[1]{\pdfcomment[author=Ben]{#1}}
\newcommand{\cc}[1]{\includegraphics[height=4mm]{img/#1.png}}
\usepackage{ifthen}
\newcommand{\license}[2][]{\\#2\ifthenelse{\equal{#1}{}}{}{\\\scriptsize\url{#1}}}
\usepackage{textcomp}

\title{Umgang mit Sozialen Netzwerken}
\author{Chaos Computer Club Dresden\\Marius Melzer, Paul Schwanse, Stephan Thamm}
\date{Stand: 22. März 2012}

\begin{document}
\maketitle

\frame{\tableofcontents[hideallsubsections]}

\section{Einleitung}
\subsection{}

\begin{frame}
    \frametitle{Motivation}
    \begin{itemize}
        \item<2-> immer tiefere Integration von Computern in das Leben der Menschen
            \note{Hörgeräte, Herzschrittmacher, Digitale Lebensaufzeichnungen, Bundestrojaner -> FAZ Feuilleton}
        \item<3-> Lücke im Bildungssystem
        \item<4-> sehen uns in der Pflicht Wissen zu vermitteln
        \item<5-> Verständnis anstatt Anleitungen
        \item<6-> Vernetzung mit Schulen und anderen Bildungseinrichtungen
    \end{itemize}
\end{frame}

\begin{frame}
    \frametitle{Wer sind wir?}
    \begin{itemize}
        \item<2-> Chaos Computer Club Dresden (\url{http://c3d2.de})
            \note{}
        \item<3-> Datenspuren (\url{http://datenspuren.de})
        \item<4-> Podcasts (\url{http://pentamedia.de})
        \item<5-> Chaos macht Schule
            \begin{itemize}
                \item \url{http://ccc.de/schule}
                \item \url{http://c3d2.de/schule.html}
            \end{itemize}
            \note{alle Folien auf einmal aufblättern? Ben's vorschlag}
    \end{itemize}
\end{frame}

\begin{frame}
    \frametitle{Chaos macht Schule}
    \begin{itemize}
        \item<2->Ziele:
            \begin{itemize}
                \item<3-> Kinder auf das Internet vorbereiten \ldots
                \item<4-> \ldots nicht das Internet auf Kinder
                    \note{Scheren-Vergleich}
                \item<5-> Informationelle Selbstbestimmung
                \item<6-> Medienkompetenz
                    \note{Medium nicht nur benutzen, sondern auch verstehen. Wir machen keinen Datenschutz-Richtlinien bei Facebook klicken Vortrag!}
                \item<7-> Kreativer Umgang mit Technik
                    \note{Eigene Dinge schaffen, weg von der Konsum-Mentalität}
            \end{itemize}
        \item<8-> Schulklassen
        \item<9-> Elternabende
        \item<10-> Lehrerfortbildung
    \end{itemize}
\end{frame}

\section{Freiheit}
\subsection{}

\begin{frame}
    \frametitle{Was wir vermitteln wollen}
    \begin{itemize}
        \item<2-> Dezentrale Dienste
            \note{man kann sich die Organisation aussuchen die seine Daten bekommt bzw. einen eigenen Server betreiben}
            \begin{itemize}
                \item<3-> Email
                \item<4-> Jabber/XMPP
                \item<5-> Diaspora, Buddycloud
            \end{itemize}
        \item<6-> Alle Sender gleichberechtigt
        \item<7-> Unix-Philosophie von Doug McIlroy:
            \begin{quote}Do one thing, do it right!
            \end{quote}
    \end{itemize}
\end{frame}

\begin{frame}
    \frametitle{Freie Lizenzen}
    \begin{itemize}
        \item<2-> Jeder ist Produzent und Konsument
        \item<3-> Urheberrecht schränkt Verwendung ein
        \item<4-> Freie Lizenzen ermöglichen Verbreitung
%      \ben{an dieser Stelle vielleicht CopyLeft als Buzzword erwähnen (daran erinnern sich dann vielleicht die Leute)}
        \item<5-> Sharing is caring
    \end{itemize}
\end{frame}

\begin{frame}
    \frametitle{Freie Medien}
    \begin{itemize}
        \item<2-> Freie Lehrmaterialien
        \item<3-> Freie Musik
            \begin{itemize}
                \item Jamendo (\url{http://www.jamendo.com/})
                \item Free Music Archive (\url{http://freemusicarchive.org/})
                \item Pentamusic (\url{http://pentamedia.org/})
            \end{itemize}
        \item<4-> Open Clip Art Library (\url{http://openclipart.org/})
        \item<5-> OpenStreetMap (\url{http://openstreetmap.de/})
    \end{itemize}
\end{frame}

\begin{frame}
    \frametitle{Freie Software}
    \begin{itemize}
        \item Linux $ \gets $ Windows
        \item Libre Office/Open Office $ \gets $ Microsoft Office
        \item Firefox $ \gets $ Internet Explorer
        \item Thunderbird $ \gets $ Outlook
        \item Gimp $ \gets $ Photoshop
        \item Inkscape $ \gets $ Illustrator
        \item VLC Media Player $ \gets $ Windows Mediaplayer
    \end{itemize}
\end{frame}

\begin{frame}
    \frametitle{Vorstellung}
\end{frame}

\begin{frame}
    \frametitle{Einführung}
    \begin{itemize}
        \item Wer ist der Meinung, schon einmal Urheberrecht verletzt zu haben?
        \item Beispiele:
        \begin{itemize}
          \item Hintergrund-Musik in einem eigenen Video
          \item Bild aus Karrikatur/Studie
          \item Austeilen einer CD mit Lied
          \item Text aus einem Arbeitsbuch kopieren
          \item Verwenden eines Wikipedia-Artikels
        \end{itemize}
    \end{itemize}
\end{frame}

\begin{frame}
    \frametitle{Motivation}
    \begin{itemize}
        \item Vorhandenes Wissen sollte verfügbar sein und weitergegeben werden dürfen
        \item Lehr- und Lernmaterialien können ohne Zugangsbeschränkungen und barrierefrei genutzt werden
        \item Materialien sollten unabhängig von Zeit und Ort genutzt werden können
        \item Jeder kann sich in den Entwicklungsprozess der Materialien einbringen
        \item Jeder kann von den Änderungen anderer profitieren
        \item neue Medien machen kollaboratives Arbeiten (erst) möglich
        \item Wissensgesellschaft
    \end{itemize}
\end{frame}

\begin{frame}
    \frametitle{Urheberrecht}
    \begin{itemize}
        \item Worauf?
        \item Grenzen/Schranken (Lehre, Forschung)
        \item Lizenzen/Nutzungsrechte
    \end{itemize}
\end{frame}
\end{document}
