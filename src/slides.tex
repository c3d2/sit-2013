\documentclass{beamer}
%\documentclass[14pt,handout]{beamer}
\usetheme{Darmstadt}
\usepackage{graphicx}
\usepackage{fixltx2e} % provides \textsubscript
\usepackage{fontspec}
\defaultfontfeatures{Mapping=tex-text,Scale=MatchLowercase}
\setmainfont{Linux Libertine O}
\setsansfont[Scale=.8]{Comic Sans MS}
\setmonofont{Adler}
\usepackage{luacode}

%\usepackage[ngerman]{babel}
\usepackage{luatextra}
\usepackage[EU2]{fontenc}

\usepackage{hyperref}
\usepackage[german]{babel}

\setbeamertemplate{footline}[frame number]

\usepackage{pdfcomment}
\newcommand{\ben}[1]{\pdfcomment[author=c3d2]{#1}}
\newcommand{\cc}[1]{\includegraphics[height=4mm]{img/#1.png}}
\usepackage{ifthen}
\newcommand{\license}[2][]{\\#2\ifthenelse{\equal{#1}{}}{}{\\\scriptsize\url{#1}}}
\usepackage{textcomp}

\title{Freie Lizenzen bei Lehr- und Lernmaterialien}
\author{Chaos Computer Club Dresden\\Marius Melzer, Paul Schwanse, Stephan Thamm}
\date{Stand: 13.03.2013}

\begin{document}
\maketitle

\frame{\tableofcontents[hideallsubsections]}

\section{Einleitung}
\subsection{}

\begin{frame}
    \frametitle{Wissen}
    \begin{itemize}
      \item Einfache Nutzung
      \begin{itemize}
        \item Verfügbarkeit ohne Zugangsbeschränkungen
        \item Unabhängig von Zeit und Raum
        \item Weiterverteilung möglich
      \end{itemize}
      \item Kollaboratives Erstellen
      \begin{itemize}
        \item Ermöglicht durch neue Medien
        \item In gemeinschaftlichem Entwicklungsprozess
        \item Profitieren von den Änderungen anderer
      \end{itemize}
    \end{itemize}
\end{frame}
 
\begin{frame}
    \frametitle{Wer sind wir?}
    \begin{itemize}
        \item<2-> Chaos Computer Club Dresden (\url{http://c3d2.de})
            \note{}
        \item<3-> Datenspuren (\url{http://datenspuren.de})
        \item<4-> Podcasts (\url{http://pentamedia.de})
        \item<5-> Chaos macht Schule
            \begin{itemize}
                \item<2-> \url{http://ccc.de/schule}
                \item<2-> \url{http://c3d2.de/schule.html}
            \end{itemize}
            \note{alle Folien auf einmal aufblättern? Ben's vorschlag}
        \item<6-> Keine Rechtsanwälte
    \end{itemize}
\end{frame}

\begin{frame}
    \frametitle{Chaos macht Schule}
    \begin{itemize}
        \item<2->Ziele:
            \begin{itemize}
                \item<3-> Kinder auf das Internet vorbereiten \ldots
                \item<4-> \ldots nicht das Internet auf Kinder
                    \note{Scheren-Vergleich}
                \item<5-> Informationelle Selbstbestimmung
                \item<6-> Medienkompetenz
                    \note{Medium nicht nur benutzen, sondern auch verstehen. Wir machen keinen Datenschutz-Richtlinien bei Facebook klicken Vortrag!}
                \item<7-> Kreativer Umgang mit Technik
                    \note{Eigene Dinge schaffen, weg von der Konsum-Mentalität}
            \end{itemize}
        \item<8-> Schulklassen
        \item<9-> Elternabende
        \item<10-> Lehrerfortbildung
        \item<11-> Keine Rechtsberatung
    \end{itemize}
\end{frame}

\begin{frame}
    \frametitle{Urheberrechtsverletzung?}
    \begin{itemize}
        \item<1-> Hintergrund-Musik in einem eigenen Video
        \item<2-> Bild aus Karrikatur/Studie
        \item<3-> Austeilen einer CD mit Lied
        \item<4-> Zeigen von Filmen im Unterricht
        \item<5-> Text aus einem Arbeitsbuch kopieren
        \item<6-> Verwenden eines Artikels aus dem Internet
    \end{itemize}
\end{frame}

\section{Urheberrecht}
\subsection{}

\begin{frame}
    \frametitle{Urheberrecht}
    \begin{itemize}
        \item<2-> Was ist Urheberrecht?
        \item<3-> Grenzen/Schranken (Lehre, Forschung,\ldots)
        \begin{itemize}
          \item<4-> Verwertungsgesellschaften
        \end{itemize}
        \item<5-> Lizenzen/Nutzungsrechte
    \end{itemize}
\end{frame}

\section{Freie Lizenzen}
\subsection{}

\begin{frame}
    \frametitle{Die 4 Freiheiten}
    \begin{itemize}
        \item<2-> Die Software:
            \begin{enumerate}
            \setcounter{enumi}{0}
                \item<3-> 0
                \item<4-> 1
                \item<5-> 2
                \item<6-> 3
            \end{enumerate}
        \item<7-> Auf Software zugeschnitten
    \end{itemize}
\end{frame}

\begin{frame}
    \frametitle{Creative Commons}
    \begin{itemize}
        \item<2-> Name der Organisation
        \item<3-> 2001 gegründet
        \item<4-> Lawrence Lessig
        \item<5-> Eine Lizenz, die ...
        \begin{itemize}
          \item<6-> eindeutig ist
          \item<7-> leicht verständlich ist
          \item<8-> Nutzung von Werken regelt
        \end{itemize}
        \item<9-> Besteht aus den Modulen BY, SA, ND, NC
    \end{itemize}
\end{frame}

%nachfolgende Logos mit einarbeiten
\begin{frame}
    \frametitle{Modul 0: BY}
    \begin{itemize}
        \item 
    \end{itemize}
\end{frame}

%by kurz erklärt
\begin{frame}
    \frametitle{Modul 1: SA}
    \begin{itemize}
        \item 
    \end{itemize}
\end{frame}

%sa kurz erklärt
\begin{frame}
    \frametitle{Modul 2: ND}
    \begin{itemize}
        \item 
    \end{itemize}
\end{frame}

%nd kurz erklärt
\begin{frame}
    \frametitle{Modul 3: ND}
    \begin{itemize}
        \item 
    \end{itemize}
\end{frame}

%nc kurz erklärt und von abgeraten!!!!!
\begin{frame}
    \frametitle{Modul 4: NC}
    \begin{itemize}
        \item 
    \end{itemize}
\end{frame}

\begin{frame}
    \frametitle{Baukastensystem}
    \begin{itemize}
        \item<2-> durch Kombination 6 Lizenzen zusammenstellbar
            %verweis https://creativecommons.org/choose/?lang=de
        \item<3-> Lizensierung des eigenen Werkes einfach möglich
        \item<4-> Fallstricke
            % http://wiki.creativecommons.org/Frequently_Asked_Questions#If_I_derive_or_adapt_a_work_offered_under_a_Creative_Commons_license.2C_which_CC_license.28s.29_can_I_apply_to_the_resulting_work.3F
    \end{itemize}
\end{frame}

\begin{frame}
    \frametitle{Kompatibilitat}
\end{frame}

\begin{frame}
    \frametitle{Weitere freie Lizenzen}
    \begin{itemize}
        \item<2-> Open Database License (ODbL)
        \item<3-> Charityware
        \item<4-> Pizzaware
    \end{itemize}
\end{frame}

\section{Projekte}
\subsection{}

\begin{frame}
    \frametitle{Open Educational Resources}
    \begin{itemize}
      \item<2-> Open Courseware:
      \begin{itemize}
        \item<3-> timm
        \item<4-> Berkley
        \item<5-> MIT
      \end{itemize}
      \item<5-> Coursera
      \item<6-> Edutags
    \end{itemize}
\end{frame}

\begin{frame}
    \frametitle{Wikimedia}
    \begin{itemize}
      \item<2-> Wikipedia
      \item<3-> Wiki Commons
      \item<4-> Wikibooks
      \item<5-> Wikiversity
      \item<6-> Wictionary
      \item<7-> Wikiquotes
      \item<8-> ... und 12 weitere
    \end{itemize}
\end{frame}

\begin{frame}
    \frametitle{CC Inhalte finden}
    \begin{itemize}
        \item<2-> Creative Commons Search
        \item<3-> Creative Commons Content Directories
        \item<4-> Google Web und Images
        \item<5-> Flickr
        \item<6-> 500px
        \item<7-> Jamendo
        \item<8-> ccMixter
        \item<9-> ...
    \end{itemize}
\end{frame}

\begin{frame}
    \frametitle{Und weiter?}
    \begin{itemize}
      \item<2-> Freie Materialien \dots
      \item<3-> \dots~in offenen Formaten \dots
      \item<4-> \dots~erstellt mit freien Werkzeugen \dots
      \item<5-> \dots~auf freien (Betriebs-)Systemen \dots
      \item<6-> \dots~und offener Hardware.
    \end{itemize}
\end{frame}

\section{Zusammenfassung}
\subsection{}

<<<<<<< HEAD
=======
\begin{frame}
    \frametitle{Fazit}
    \begin{itemize}
    \end{itemize}
\end{frame}

\begin{frame}
    \frametitle{Diskussion}
\end{frame}

>>>>>>> 8423650523ebb37aa5f3288fc29c8986994fb6cc
\end{document}
